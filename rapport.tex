\documentclass[a4paper,12pt,twoside]{report}
\usepackage{xcolor}
\usepackage[french]{babel}  % Pour le français
\usepackage[utf8]{inputenc} % Pour taper les caractères accentués

\usepackage{amsmath}  % Ces trois paquets donnent accès à 
\usepackage{amsfonts} % des symboles et formulations
\usepackage{amstext}  % mathématiques
\usepackage{hyperref} % Permet de faire automatiquement des liens dans les
                      % documents
\usepackage{graphicx} % Permet d'insérer des images
\DeclareGraphicsExtensions{.png} % Mettre ici la liste des extensions des
                                 % fichiers images

% On peut choisir la police en utilisant un paquet 
%\usepackage{newcent}
\usepackage{lmodern}
%\usepackage{cmbright} % Computer Modern Bright

% Une des nombreuses manière de modifier les marges par défaut
\usepackage{geometry}
\geometry{vmargin=2cm,hmargin=2.5cm,nohead}

% On peut redéfinir certaines longueurs, par exemple l'espacement entre les
% paragraphes:
\setlength{\parskip}{0.25cm}

% Quelques définitions 
\def \rr {{\mathbb R}} % L'ensemble R
\def \cc {{\mathbb C}} % L'ensemble C
\def \nn {{\mathbb N}} % L'ensemble N
\def \zz {{\mathbb Z}} % L'ensemble Z

% Les informations de la page de titre (page de titre séparée pour un 'report').
\title{Rapport sur les méthodes de décomposition de domaine }
\author{
 Armel
  \and
Nayir \and Leila
}
% \thanks: permet de mettre une note de bas de page pour l'auteur
% \href: insère un lien, ici vers l'application 'mailto'
% \tt: police monospace
\date{\today} % \today pour la date courante 

\begin{document}

\maketitle % Page de titre automatique à partir des infos ci-dessus

% La commande cleardoublepage est utilisée pour s'assurer que la page suivante
% est une page de droite lorque l'on imprime recto-verso.
\cleardoublepage
\tableofcontents % Table des matière automatique à partir des chapitres,
                 % sections, etc du document

\cleardoublepage
\chapter{Introduction }
Les méthodes de décomposition de domaine de type Shwarz consistent à remplacer la résolution d'une équation différentielle sur un domaine par la résolution de cette même équation sur une succession de sous-domaines plus petits, plus simples le composant puis par assemblage des solutions respectives en une seule qui constitue l'approximation de la solution générale.

Ici, l'objectif est d'étudier la convergence des algorithmes de méthodes de décomposition de domaine pour la résolution de l'équation elliptique linéaire, à coefficients constants avec conditions aux limites de types Dirichlet homogène suivante :

\[
\left\{
\begin{array}{r c l}
 -\Delta u + \alpha u   &=& f \quad dans \quad \Omega  \\
 u&=&0 \quad dans \quad \partial\Omega\\
\end{array}
\right.
\]

% les espacements entre paragraphes, indentation, etc sont automatiques.
Pour cela la méthode des différences finies est utilisée.

\section{Méthode des différences finies
}
L'opérateur laplacien est approximé par la méthode des différences finies par l’expression $ \frac{2u_{i}-u_{i-1} -u_{i+1}}{h^{2}} $. En effet, en remplaçant les différents termes par leurs approximations respectives de Taylor, le laplacien est retrouvé.

En \textbf{\color[rgb]{0,0.13,0.75}1D}:

Si le domaine est [a,b], N+2 points le nombre total du domaine, $ h=\frac{1}{ N+1} $   le pas de discrétisation, $ x_{i}=ih$ ,$ f_{i}=f(xi)$, $u_{o}=u(a)$ et $u_{n+1}=u(b)$, la solution étant recherchée  à l'intérieur du domaine, le schéma aux différences finies fournit les systèmes suivants pour i=1,..,N:


\[
\left\{
\begin{array}{r c l}
 \frac{2u_{i}-u_{i-1} -u_{i+1}}{h^{2}}  + \alpha u_{i }&=& f_{i}  \\
 u_{0}&=&u(a)\\
u_{n+1} &=& u(b)
\end{array}
\right.
\]


\[
\begin{pmatrix}
   \frac{2}{h^{2}} +\alpha & \frac{-1}{h^{2}}  &0 &\cdots & 0 \\
   \frac{-1}{h^{2}}  & \frac{2}{h^{2}}  +\alpha & \frac{-1}{h^{2}} & \ddots &  \vdots \\ 0  & \ddots &  \ddots &  \ddots & 0\\
   \vdots & \ddots & \ddots  & \frac{2}{h^{2}} +\alpha &  \frac{-1}{h^{2}}\\
 0 & \cdots & 0 & \frac{-1}{h^{2}} & \frac{2}{h^{2}} +\alpha
\end{pmatrix}
\begin{pmatrix}
  u_{1} \\
  u_{2}\\ 
   \vdots\\
   \vdots\\
 u_{N}
\end{pmatrix} = \begin{pmatrix}
  f_{1} \\
   f_{2} \\ 
   \vdots\\
   \vdots\\
f_{N}
\end{pmatrix} +  \begin{pmatrix}
   \frac{u_{0}}{h^{2}} \\
   0 \\ 
   \vdots\\
   \vdots\\
 \frac{u_{N+1}}{h^{2}} 
\end{pmatrix} \]


Les éléments de la diagonales sont strictements positifs, la matrice est symétrique à diagonale fortement dominante et irréductible donc le système est inversible et le système admet une unique solution.

En \textbf{\color[rgb]{0,0.13,0.75 }2D}, l'opérateur laplacien est approximé par : 

\[
\left\{
\begin{array}{r c l}
\frac{1}{h^{2}}(4u_{ij} -u_{i+1,j} -u_{i-1,j}-u_{i,j+1} -u_{i,j-1})=f_{ij} \quad pour \quad 1 \leq i,j \leq n  \\
u_{ij}=0 \quad pour\quad i=0\quad ou \quad i=n+1\quad ou \quad j=0\quad ou \quad j=n+1\\
\end{array}
\right.
\]
	En \textbf{\color[rgb]{0,0.13,0.75}3D}, par :
\[
\left\{
\begin{array}{r c l}
\frac{1}{h^{2}}(-u_{i+1,kl} - u_{i,k+1,l} -u_{i,k,l+1} + 6_{ui,k,l} -u_{i,k,l-1} - u_{i,k-1,l} -u_{i-1,k,l}) = f_{i,k,l} \quad pour  \quad \leq i,k,l\leq n   \\
u_{i,k,l}=0  \quad pour \quad  i=0 \quad  ou \quad    i=n+1 \quad   ou \quad  k=0 \quad  ou \quad  k=n+1\quad   ou \quad  l=0 \quad  ou \quad  l=n+1 
\\
\end{array}
\right.
\]






\textbf{Exemples :
}

En \textbf{\color[rgb]{0,0.13,0.75}
 1D} avec \textbf{$N=100:$}

En imposant comme solution $u(x)=sin(x{\pi}^2)$, \textbf{ f(x) } peut être déduit analytiquement et vaut :$ (1+{\pi}^2)sin(x\pi) $ et l'approximation par le schéma aux différences finies permet de retrouver la solution avec une erreur qui diminue lorsque le pas de discrétisation \textbf{h} diminue:
\begin{figure}[htpb!]
\includegraphics[width=0.6\linewidth]{11.png}
  \includegraphics[width=0.6 \linewidth]{10.png}

\end{figure}


\












\clearpage

De même en \textbf{\color[rgb]{0,0.13,0.75}
2D,} avec \textbf{$N= 100$ } dans les deux dimensions et en imposant $u(x,y)=sin(x{\pi})sin(y{\pi})$, $ f(x,y)=(1+2{\pi}^2)sin(x{\pi})sin(y{\pi})$:
\begin{figure}[htpb!]
 
\includegraphics[width=0.6\linewidth]{12.png} 
  \includegraphics[width=0.6\linewidth]{Figure_1-3.png}
  \label{fig:exemple}
\end{figure}


\

En \textbf{\color[rgb]{0,0.13,0.75}
3D,} pour \textbf{N=100} dans les trois dimensions et en imposant $u(x,y,z)=sin(x{\pi})sin(y{\pi})sin(z{\pi})$,  $ f(x,y,z)=(1+3{\pi}^2)sin(x{\pi})sin(y{\pi})sin(z{\pi})$

\begin{tabular}{|l|c|c|c|r|}
  \hline
  Taille & hx  & hy  & hz & Erreur absolue \\
  \hline
  27000 & 3.22581e-02   & 3.22581e-02& 3.22581e-02&5.05450e-02  \\
  \hline
\end{tabular}

\cleardoublepage

\chapter{Quelques méthodes de décomposition de domaines}


Plusieurs méthodes de décomposition existent. Pour différencier celles-ci, en prenant par exemple le cas de la \textbf{dimension 1} et en décomposant un domaine simple type \textbf{[a,b]} en deux sous-domaines 
\textbf{\color[rgb]{0,0.58,0}
[a,d]} et \textbf{\color[rgb]{1,0,0}
[c,b]} comme ci -dessous, la résolution successive de l'équation sur ces deux sous-domaines  avec leurs conditions aux limites respectives se fait de la manière suivante :
une première solution est recherchée sur \textbf{[a,d] }pour cela une condition aux limites initiales est imposée $ u(d)=\mu
  $


\[
\left\{
\begin{array}{r c l}
 -\Delta u + \alpha u_{0} &=&f \quad  dans\quad [a,d] \\
 u_{0}(a)&=&0\\
u_{0}(d)&=&\mu
\end{array}
\right.
\]



Puis, pour tout entier p positif, une solution $u_{\color[rgb]{1,0,0}2p-1} $ est recherchée dans \textbf{[c,b]} et $u_{\color[rgb]{0.2,0.8,0.2}
2p}$ dans \textbf{[a,d]}. Chacune ayant des conditions aux limites différentes :  la condition aux limites en (c) de $u_{\color[rgb]{1,0,0}2p-1}$ est fournie par la valeur de $u_{\color[rgb]{0.2,0.8,0.2}
2p-2}$ au même endroit tandis que la condition aux limites en (d) de $u_{\color[rgb]{0.2,0.8,0.2}
2p}$ est fournie par la valeur de $u_{\color[rgb]{1,0,0}2p-1}$ au même endroit. Ainsi les conditions aux limites de $u_{\color[rgb]{1,0,0}2p-1}$ (resp $u_{\color[rgb]{0.2,0.8,0.2}
2p}$) sont $u_{\color[rgb]{1,0,0}2p-1}(c)= u_{\color[rgb]{0.2,0.8,0.2}
2p}(c) $
$et   $ $ u_{\color[rgb]{1,0,0}2p-1}(b)=0$ (resp $u_{\color[rgb]{0.2,0.8,0.2}
2p}(a)=0$ et $u_{\color[rgb]{0.2,0.8,0.2}
2p}(d)= u_{\color[rgb]{1,0,0}2p-1}(d)$).


La méthode est dite alternée si la condition aux limites $u_{\color[rgb]{1,0,0}2p-1}(c)$ s'obtient par la valeur de $u_{\color[rgb]{0.2,0.8,0.2}
2p}(c)$  de l'itération précédente, c'est-à-dire $u_{\color[rgb]{0.2,0.8,0.2}
2p-2}(c)$ et celle de $u_{\color[rgb]{0.2,0.8,0.2}
2p}(d)$  s'obtient par la valeur de $u_{\color[rgb]{1,0,0}2p-1}(d)$ de cette même itération. Il s'agit donc de résoudre le système suivant :
\[
\left\{
\begin{array}{r c l}
 -\Delta u_{2p-1} + \alpha u_{2p-1} &=&f \quad  dans\quad [c,b] \\
 u_{2p-1}(c)&=& u_{2p-2}(c)\\
 u_{2p-1}(b)&=&0\\
-\Delta u_{2p} + \alpha u_{2p} &=&}f \quad dans \quad [a,d] \\  u_{2p}(a)&=& 0\\ u_{2p}(d)&=&u_{2p-1}(d)
\end{array}
\right.
\]


Il existe aussi une méthode dite parallèle arrivée plus tardivement  où, cette fois-ci, la condition aux limites $u_{\color[rgb]{0.2,0.8,0.2}2p}(d)$ est obtenue par $u_{\color[rgb]{1,0,0}2p-1}(d)$ de l'itération précédente. 
\clearpage

Pour résumer:
\begin{center}
Scharz alterné

\end{center}
\[
\left\{
\begin{array}{r c 2}
 Lu_{1}^{n+1} &=&f  \quad  dans\quad \Omega 1  \\
 u_{1}^{n+1}&=&u_{2}^{n} \quad\ sur \quad \partial\Omega1 \\
 Lu_{2}^{n+1}&=&f dans \quad \Omega 2 \\ 
{\color[rgb]{1,0,0}
u_{2}^{n+1}&=&{\color[rgb]{1,0,0} u_{1}^{n+1} }\quad sur \quad \partial\Omega2\\
\end{array}
\right.
\]
	\begin{center}
Scharz parallèle

\end{center}
\[
\left\{
\begin{array}{r c 2}
 Lu_{1}^{n+1} &=&f  \quad  dans\quad \Omega 1  \\
 u_{1}^{n+1}&=&u_{2}^{n} \quad\ sur \quad \partial\Omega1 \\
 Lu_{2}^{n+1}&=&f dans \quad \Omega 2 \\ 
{\color[rgb]{1,0,0}
u_{2}^{n+1}&=& {\color[rgb]{1,0,0}u_{1}^{n}} \quad sur \quad \partial\Omega2\\
\end{array}
\right.
\]


 


\chapter{Application à la résolution d'une équation différentielle}
 
 Ici, la méthode alternée est implémentée.

Soit N le nombre de points du domaine \textbf{[a,b]}, $h=\frac{1}{ N+1} $   le pas de discrétisation.
$c=kh$ et $ d=qh$. k et q étants deux entiers tels que k soit inférieur strictement à q, le recouvrement est donc r=[kh,qh].
Soit \textbf{n2} le nombre de points intérieurs à \textbf{[a,d]} et \textbf{m} celui de \textbf{[c,b] }donc \textbf{n2=q-1}. 

Si p=0, le système à résoudre pour $u_{0}$ est :

\[
\begin{pmatrix}
   \frac{2}{h^{2}} +\alpha & \frac{-1}{h^{2}}  &0 &\cdots & 0 \\
   \frac{-1}{h^{2}}  & \frac{2}{h^{2}}  +\alpha & \frac{-1}{h^{2}} & \ddots &  \vdots \\ 0  & \ddots &  \ddots &  \ddots & 0\\   \vdots & \ddots & \ddots  & \frac{2}{h^{2}} +\alpha &  \frac{-1}{h^{2}}\\
 0 & \cdots & 0 & \frac{-1}{h^{2}} & \frac{2}{h^{2}} +\alpha
\end{pmatrix}
\begin{pmatrix}
  u_{0}^{1} \\u_{0}^{2}\\  \vdots\\ \vdots\\ u_{0}^{n}
\end{pmatrix} = \begin{pmatrix}
  f_{1} \\ f_{2} \\ \vdots\\\vdots\\f_{N}
\end{pmatrix} +  \begin{pmatrix}0\\ 0 \\   \vdots\\
   \vdots\\
 \frac{\mu}{h^{2}} 
\end{pmatrix} \]
Si  p est positif, le système à résoudre pour 
$u_{\color[rgb]{1,0,0}2p-1} $ est :

\[
\begin{pmatrix}
   \frac{2}{h^{2}} +\alpha & \frac{-1}{h^{2}}  &0 &\cdots & 0 \\
   \frac{-1}{h^{2}}  & \frac{2}{h^{2}}  +\alpha & \frac{-1}{h^{2}} & \ddots &  \vdots \\ 0  & \ddots &  \ddots &  \ddots & 0\\   \vdots & \ddots & \ddots  & \frac{2}{h^{2}} +\alpha &  \frac{-1}{h^{2}}\\
 0 & \cdots & 0 & \frac{-1}{h^{2}} & \frac{2}{h^{2}} +\alpha
\end{pmatrix}
\begin{pmatrix}
  u_{2p-1}^{1} \\u_{2p-1}^{2}\\  \vdots\\ \vdots\\ u_{2p-1}^{m}
\end{pmatrix} = \begin{pmatrix}
  f_{1+k} \\ f_{2+k} \\ \vdots\\\vdots\\f_{m+k}
\end{pmatrix} +  \begin{pmatrix}\\\frac{u_{2p-2}(c)}{h^{2}}\\ 0 \\   \vdots\\
   \vdots\\
  0
\end{pmatrix} \]
et pour ${u_{\color[rgb]{0,0.58,0}2p}}$:

\[
\begin{pmatrix}
   \frac{2}{h^{2}} +\alpha & \frac{-1}{h^{2}}  &0 &\cdots & 0 \\
   \frac{-1}{h^{2}}  & \frac{2}{h^{2}}  +\alpha & \frac{-1}{h^{2}} & \ddots &  \vdots \\ 0  & \ddots &  \ddots &  \ddots & 0\\   \vdots & \ddots & \ddots  & \frac{2}{h^{2}} +\alpha &  \frac{-1}{h^{2}}\\
 0 & \cdots & 0 & \frac{-1}{h^{2}} & \frac{2}{h^{2}} +\alpha
\end{pmatrix}
\begin{pmatrix}
  u_{2p}^{1} \\u_{2p}^{2}\\  \vdots\\ \vdots\\ u_{2p}^{n2}
\end{pmatrix} = \begin{pmatrix}
  f_{1} \\ f_{2} \\ \vdots\\\vdots\\f_{n2}
\end{pmatrix} +  \begin{pmatrix}\\0\\ 0 \\   \vdots\\
   \vdots\\
  \frac{u_{2p-1}(d)}{h^{2}}
\end{pmatrix} \]


La solution approchée de l'équation sur l'ensemble du domaine est obtenu en assemblant les \textbf{n2} (=q-1) composantes du vecteur $u_{2p} $ avec les {\textbf{n-q+1} } dernières composantes de $u_{2p-1}$ (ce qui fait \textbf{n} valeurs en tout comme le vecteur de la solution exacte qui est à n composantes). La convergence est obtenue si le terme $\frac{|U_{exact}-U_{approché}|}{|U_{exact}|} $  est inférieur à un certain $\epsilon$.
 

u=(u_{\color[rgb]{0,0.58,0}
2p}(1),u_{\color[rgb]{0,0.58,0}
2p}(2), ...., u_{\color[rgb]{0,0.58,0}
2p}(n2), u_{\color[rgb]{1,0,0}
2p-1}(q-k), u_{\color[rgb]{1,0,0}
2p-1}(q-k+1),..u_{\color[rgb]{1,0,0}
2p-1}(m))


\section{Algorithme}

\begin{figure}[htpb!]
  \includegraphics[width=0.9\linewidth]{pg1.JPG} 
    \includegraphics[width=0.9\linewidth]{pg2.JPG}
  \caption{Algorithme de décomposition de domaine en dimension 1}
  \label{fig:exemple}
\end{figure}


 \section{Preuve de la convergence en 1D}
}
Soit $\alpha =0, a=0, b=1, u(a)=u(b)=0$ et $ f=0. $ L'équation devient  $-\Delta u =0 $ et donc $u=qx + r$ et comme $ u(a)=u(b)$, $u_{exact}=0$


\[
\left\{
\begin{array}{r c 1}
 -\Delta u_{2p-1} &=&0 \quad  dans\quad (c,b) \\
 u_{2p-1}(c)&=&u_{2p-2}(c )\\
  u_{2p-1}(b)&=&0
\end{array}
\right.
\]

$-\Delta u_{2p-1} =0$  entraîne  $u_{2p-1} =q_{2p-1}x + r_{2p-1} $
De même $u_{2p} =q_{2p}x + r_{2p} $.
 $u_{2p}(a) =u_{2p}(0)=q_{2p} x 0+ r_{2p}=   r_{2p} $et puisque $ u_{2p}(a) =0 $ alors $r_{2p}=0 $  donc $ u_{2p}(x)= q_{2p}x $
$ u_{2p-1}(b) =q_{2p-1}x+ r_{2p} $ et puisque $  u_{2p}(b)=0, r_{2p-1}=-q_{2p-1}$
 donc $u_{2p-1}(x)=q_{2p-1}(x-1)$
 
$ u_{2p-1}(c)=u_{2p-2}(c) $ donc $ q_{2p-1}(c-1)=q_{2p-2}(c) $ et de même $q_{2p}(d)=q_{2p-1}(d-1)$
 Ce qui entraîne $ q_{2p}(d)=q_{2p-2}x \frac{c}{c-1} x (d-1)$ et $q_{2p}=\frac{c(1-d)}{d(1-c)}xq_{2p-2} $ soit $u_{2p}(x)=\frac{c(1-d)}{d(1-c)}x u_{2p-2} (x)$
De même, $ u_{2p-1}(x)=\frac{c(1-d)}{d(1-c)}xq_{2p-3}(x) $ et $ u_{2p-1} $ converge vers 0 et l'algorithme converge.  Cette convergence est géométrique de raison  $\frac{c(1-d)}{d(1-c)} $et plus le recouvrement\textbf{ [c,d]} est grand plus la raison diminue et la convergence est rapide mais si le recouvrement est nul l'algorithme est stationnaire et ne peut converger.
}


\section{Approximation de la solution dans le cas 1D}

Pour $\textbf{N=100}$ , le nombre de points du maillage,  \textbf{$\epsilon=0.0001$}, le seuil de convergence, l'algorithme fournit une solution approchée.
\begin{figure}[htpb!]
\centering
  \includegraphics[width=0.8\linewidth]{Figure_2.png}
  \caption{Solution exacte et solution approchée par la méthode de décomposition de domaine}
  \label{fig:exemple}
\end{figure}


\begin{figure}[htpb!]
\centering
  \includegraphics[width=0.8\linewidth]{Figure_21.png}  \caption{Nombre d'itérations nécessaires à la convergence en fonction de la taille du recouvrement.}
  \label{fig:exemple}
\end{figure}
Plus le recouvrement est gros \textbf{r=[kh,qh]}, plus le nombre d'itérations nécessaire à la convergence diminue.





\section{Approximation de la solution dans le cas 2D}
Pour \textbf{$N=100$} dans les deux dimensions, (le pas de discrétisation est donc le même dans les deux dimensions) et en imposant $u(x,y)=sin(x\pi)sin(y\pi)$, \textbf{f(x,y)} est déduit analytiquement et vaut $(1+2\pi^{2}
)sin(x\pi)sin(y\pi)$.
et la méthode de décomposition de domaine appliquée à la résolution de l'équation ci-dessus fournit une solution approchée.
\begin{figure}[htpb!]
  \includegraphics[width=0.5\linewidth]{Figure_1-3.png} 
    \includegraphics[width=0.5\linewidth]{35.png}
  \caption{Solution exacte et solution approchée par la méthode de décomposition de domaine}
  \label{fig:exemple}
\end{figure}\


\section{Approximation de la solution dans le cas 3D}
Pour \textbf{$N=30$}\footnote{\textit{Le nombre de points de maillage étant volontairement diminué car les calculs sont nombreux et lents.}} dans les trois dimensions et avec cette fois-ci, un recouvrement de 6 points seulement et une précision de 0.001  et en imposant $u(x,y,z)=sin(x\pi)sin(y\pi)sin(z\pi)$ et $f(x,y,z)= (1+3\pi^{2}
)sin(x\pi)sin(y\pi)sin(z\pi)$, l'algorithme fournit l'approximation recherchée.



\begin{tabular}{|l|c|c|c|r|}
  \hline
  Taille  & hx & hy  & hz & Erreur absolue \\
  \hline
  27000 & 3.22581e-02   & 3.22581e-02& 3.22581e-02&4.40327e-02  \\
  \hline
\end{tabular}

\cleardoublepage





  

\bibliographystyle{plain}
\bibliography{refs}

\textbf{Bibliographie}

FOUNE Babacar, Décomposition de domaine
\end{document}